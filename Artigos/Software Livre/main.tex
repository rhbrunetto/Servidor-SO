% Pacotes
\documentclass[12pt]{article}
\usepackage{adjustbox}
\usepackage[utf8]{inputenc}
\usepackage{amsmath}
\usepackage{hyperref}
\usepackage{sbc-template}
\usepackage{fancyvrb}
\usepackage{amsfonts}
\usepackage{amsmath}
\usepackage{graphicx,url}
\usepackage{float}
\usepackage{indentfirst}


\sloppy

\title{Atom\\O Editor Open Source do GitHub}

\author{Ricardo Henrique Brunetto\inst{1}}


\address{Departamento de Informática -- Universidade Estadual de Maringá (UEM)\\
	Maringá -- PR -- Brasil
	\email{ra94182@uem.br}
}

\begin{document}

	\maketitle

	{\resumo{O presente trabalho visa a apresentar de forma instrutiva as funcionalidades do editor de texto Atom e tratar a respeito de suas vantagens, desvantagens, formas de instalação e recursos que o diferenciam dos demais editores. Este trabalho segue referência direta da documentação oficial do Atom\cite{doc:atom}.}}

  \section{Introdução}
	O Atom é um editor de texto de código aberto desenvolvido pelo GitHub e liberado como beta em junho de 2015, projetado para combinar flexibilidade e extensibilidade. De forma geral, o Atom é um editor focado em desenvolvimento de código, oferecendo recursos e ferramentas para solucionar dos mais ínfimos aos mais significativos inconvenientes do ofício. Além disso, por ser um software livre e aberto e fornecer suporte à extensões criadas pelos própriros usuários, o Atom passou a ser conhecido como "Editor de Texto Hackeável do Século 21".

	O Atom foi desenvolvido através do Electron e de outras tecnologias web (HTML, Javascript e CSS).

	\bibliographystyle{sbc}
	\bibliography{references}

\end{document}
