% Pacotes
\documentclass[12pt]{article}
\usepackage{adjustbox}
\usepackage[utf8]{inputenc}
\usepackage{amsmath}
\usepackage{hyperref}
\usepackage{sbc-template}
\usepackage{fancyvrb}
\usepackage{amsfonts}
\usepackage{amsmath}
\usepackage{graphicx,url}


\sloppy

\title{Atom\\O Editor Open Source do GitHub}

\author{Ricardo Henrique Brunetto\inst{1}}


\address{Departamento de Informática -- Universidade Estadual de Maringá (UEM)\\
	Maringá -- PR -- Brasil
	\email{ra94182@uem.br}
}

\begin{document}

	\maketitle

	{\resumo{O presente trabalho visa a apresentar de forma instrutiva as funcionalidades do editor de texto Atom e tratar a respeito de suas vantagens, desvantagens, formas de instalação e recursos que o diferenciam dos demais editores. Este trabalho segue referência direta da documentação oficial do Atom \cite{doc:atom}.}}

  \section{Introdução}
	O Atom é um editor de texto de código aberto desenvolvido pelo GitHub e liberado como beta em junho de 2015, projetado para combinar flexibilidade e extensibilidade. De forma geral, o Atom é um editor focado em desenvolvimento de código, oferecendo recursos e ferramentas para solucionar dos mais ínfimos aos mais significativos inconvenientes do ofício. Além disso, por ser um software livre e aberto e fornecer suporte à extensões criadas pelos própriros usuários, o Atom passou a ser conhecido como "Editor de Texto Hackeável do Século 21".

	O Atom foi desenvolvido através do Electron e de outras tecnologias web (HTML, Javascript e CSS). O ponto-chave do Atom como editor de texto é um compromisso com a hackeabilidade e usabilidade, o que significa que os principais avanços e diferenciais do software se concentram em proporcionar ao usuário uma forma de personalizar por completo sua experiência. Dessa forma, o usuário consegue criar \textit{plug-ins} e extensões conforme sua necessidade e propósito.

	Dessa forma, o Atom é composto por seu núcleo, ferramentas e componentes que são adotados como oficiais e vêm instalados (bem como o próprio editor), e pelos \textit{add-ons} desenvolvidos pelos próprios usuários e disponíveis em repositórios na web (em especial no GitHub). A respeito do núcleo do Atom, alguns aspectos são interessantes e passíveis de abordagem.

	Duas décadas de desenvolvimento Web permitiram que a mesma evoluísse para uma incrível e poderosa plataforma. Contudo, codificar é uma tarefa especial que requer ferramentas dedicadas. Por isso, o Atom não foi escrito como uma aplicação web tradicional, mas sim como uma variante específica do Chromium, dedicada à escrita de texto.

	Abrindo um rápido parênteses, o Chromium é um conjunto de projetos que incluem um navegador web de código aberto e livre que a Google usa como base para o desenvolvimento do Google Chrome, e um sistema operacional onde a Google também se baseia para o Chrome OS. Mais informações podem ser encontradas em \cite{doc:chromium}.

	Outro grande benefício é garantir que tudo está rodando na mais nova versão do Chromium. Isso significa que não há preocupações em relação à compatibilidade ou versionamentos. Dessa forma, os mais recentes recursos e frameworks desenvolvidos na Web para uso em aplicações podem ser incluídos no Atom sem que haja problemas de compatibilidade com projetos já em desenvolvimento.

	Desenvolver um editor de texto baseado em tecnologias Web é um acerto no sentido em que se tem grande capacidade de crescimento, visto que, embora as tecnolgias nativas variem entre si, as tecnologias Web permanecem por serem multi-plataforma e irrestritas quanto às possiblidades de desenvolvimento.

	De acordo com os próprios desenvolvedores, o Atom exerce um papel complementar à função do GitHub de proporcionar software melhor através do trabalho em equipe, o que implica que o editor é, na verdade, um investimento à longo prazo, sempre respaldado pelo suporte do próprio GitHub. Além disso, conforme outros editores como Emacs e Vim demonstraram com o passar dos anos, o desenvolvimento de um software estável, de grande comunidade e eficiente, necessita ter código aberto.

	\section{Instalação}

	\section{Uso do Atom}

	\section{Hackeando o Atom}

	\section{Conclusão e Análise}
	Dentre as análises

	\bibliographystyle{sbc}
	\bibliography{references}

\end{document}
